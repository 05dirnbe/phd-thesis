%!TEX root = ../../../thesis.tex

\chapter{Continued production of dried sclerotia}

	This document contains information supplementing the main manuscript of \emph{Introducing the Slime Mold Graph Repository}.

	Here we describe how to continuously produce dried sclerotia~\cite{lifecycle} from a small initial sample\footnote{Samples of dried P. polycephalum sclerotia can be purchased online or shared via the \emph{P.~polycephalum} community, e.g. at the~\href{http://slimoco.ning.com/}{Slime Mold Collective}.} of the same. First, we prepare a $22$~cm~$\times$~$32$~cm plastic dish with a layer of $1 \%$~Agar (Kobe I) and line up a generous amount of sclerotia evenly along the short side of the dish, forming an inoculation line. Ideally, the dish is kept in a light-proof box with high humidity at a temperature around $\SI{22}{\degree}$. That being said, forest-dwelling \emph{P.~polycephalum} is rather forgiving and will make do with room-temperature and room-humidity provided it is properly shielded from light.

	After $10-14$ hours \emph{P.~polycephalum} changes from the sclerotia state to the plasmodium stage and starts to explore the dish. From this point on we recommend to feed it with oat flakes every $4-6$~hours. To do so, distribute a small amount of oat flakes evenly across the inoculation line. Given the additional nutrients the organism will rapidly cover the whole dish at a rate of approximately $1$~cm per hour. Shortly before it reaches the far side of the dish we proceed to transfer the organism to new containers. The mass covering the oat-flakes is moved to a new dish with ample room to keep the growth of the plasmodium going in a continuous fashion. The remainder of the organism is transfered to a bucket allowing it to dry, thus triggering the transition back to the stage of sclerotia. A detailed description of both steps follows.

	First, we move the plasmodium at the inoculation line together with the accumulated overgrown oat-flakes to a new $22$~cm~$\times$~$32$~cm dish prepared with agar gel. To do so we cut the agar of the old dish underneath the inoculation line into suitable pieces, which we transfer to the new dish using a small spatula. We arrange the pieces of gel carrying the plasmodium akin to the inoculation line. Soon the plasmodium will proceed to conquer the new dish. From time to time we add a small amount of oat flakes to the starting line to make sure \emph{P.~polycephalum} keeps growing steadily. Transferring to a new dish serves to keep the growth of the organism going continuously. This step can be omitted if continued production of plasmodium is not desired. If at some point the growth of the slime mold seems to have come to an unintentional halt, we recommend to make sure it is not to dry by carefully moistening \emph{P.~polycephalum} using distillate water and a plant sprayer.

	Next, we move the remaining contents of the overgrown gel to a $40$~cm high and $25$~cm wide ordinary cylindrical plastic bucket. In particular we make sure to transfer the whole growing front since it constitutes a significant part of the biomass of the organism. We cover the bottom of the bucket with tissue paper and stick moist filter paper to the walls of the bucket such that its inside is completely covered. After a while the plasmodium starts exploring the bucket and is naturally drawn towards the moistened areas. Eventually it will move away from the drier bottom and move along the bucket walls covering the filter paper. As soon as most of the filter paper is occupied by \emph{P.~polycephalum}, the paper can be removed, wrapped up and stored to dry triggering the transition from plasmodium to sclerotia. Letting the wrapped up filter paper dry in a cardboard box is simple and turned out to be effective. After approximately $3$ days, the cell mass is fully dry and in the state of sclerotia. They can easily be handled and stored for later use. Once dried, the sclerotia can be scraped off the filter paper and used with precision in the main experiments.
