%!TEX root = ../../../thesis.tex

\chapter{Details of Data Acquisition}\label{app:smgr}

	Let us now explain all steps involved in furnishing the \data in full detail. First, we describe how to setup and execute necessary wet-lab experiments, including the production of sclerotia~\cite{lifecycle}. Next, we explain how to turn the network structures depicted in the raw images into series of equivalent graphs. Finally, we illustrate how to establish unique node identities and track them within a given series of graphs.

	\section{Experiments}

		For our experiments we cultivate \P (HU195xHU200) in a rectangular, \SI{20 x 30 x 13}{\centi\metre}, translucent plastic dish ontop of a \SI{10}{\milli\metre} layer of \SI{1.25}{\percent} agar (Kobe I). To do so we place \SI{1.5}{\gram} of dried \P sclerotia crumbs along the short edge of the dish. We make sure to evenly spread them out such that a continuous and straight line is formed, connecting two adjacent edges of the dish. In the following we refer to this line as the \emph{inoculation line}, see \Fref{fig:sequence_1}. This concludes the preparation of the dish. 

		Since \P is sensitive to light, we place the dish inside a large light-proof wooden box of \SI{110 x 110 x 110}{\centi\metre}. Temperature and humidity inside were kept constant at \SI{22}{\celsius} and \SIrange{55}{60}{\percent} relative humidity. In our setup we rely on dried sclerotia, rather than plasmodium, because the former give exact control over the initial mass of \P introduced to the dish. We make sure to keep the input masses, the properties of the agar layers and the environment constant to ensure consistent repetition of experiments. For a detailed description on how to produce dried sclerotia from an initial sample we refer the reader to \Fref{sec:sclerotia}.

		Inside the box we fix a digital camera (Canon EOS 645D, Lens EFS \SIrange{18}{55}{\milli\metre}) \SI{16}{\centi\metre} above the dish. The camera is oriented perpendicular to the dish and centered right above it. Each shot captures a large area of \SI{10 x 15}{\centi\metre} at a resolution of \SI{5184 x 3456}{\pixel} in JPG format. With these settings \SI{1}{\centi\metre} on the dish corresponds to \SI{370.625}{pixel} in the image. Since during graph extraction all lengths and widths measured are stored in units of \si{\pixel}, this information can be used to map pixels back to centimeters.

		To provide the necessary light for the camera to work inside the dark box we opt for bright field illumination using a negatoscope, also known as X-ray film viewer (Planilux, $2 \times \SI{15}{\W}$, emitting white light). It provides a large area of low intensity illumination which is uniform in space and time. By putting the translucent dish ontop of the negatoscope the light that passes trough makes the structures formed by \P visible to the camera overhead. By design we ensure optimal contrast between the networks and the background and eliminate all sources of reflections or shadows in the images. This is particularly desirable as such effects are diminishing the effectiveness of \NEFI. This concludes the preparation of the box. A schematic of the complete setup can be seen in \Fref{fig:setup}.

		After the prepared dish is placed in the box, it takes roughly \SI{15}{\hour} for the organism to make its transition from sclerotia to plasmodium. Once the plasmodium begins to spread towards the far side of the dish we start capturing its growth progress by taking an image every \SI{120}{\second} using dedicated software (Motion detection software; Vulpessoft, DSLR Master). We stop capturing when the growing front first hits the adjacent wall of the dish. By doing so we minimize the probability of \P moving back towards the inoculation line. We do not feed the organism throughout the entire experiment. This concludes one iteration of our experiments.

		We repeat this experiment under constant conditions and obtain $81$~image series depicting the growth of \P and the networks it forms. Since there is a natural variability in the growth of the organism, we obtain series of different length and nature. We refer to this data as raw data. It is available in the \SMGR.

		We are aware of a potential caveat of our approach, namely the light source in the negatoscope emitting the full spectrum of white light. It is well known that \P reacts to specific parts of the spectrum while it is insensitive to others~\cite{nakagaki1996action}. Thus, ideally one chooses a light source such that the organism remains undisturbed. However, such a light source was not at our disposal so we decided to minimize the impact of the light by minimizing the time \P is exposed to it. In particular we couple the triggering of the camera with the power supply of the negatoscope. Thus we ensure that the slime mold is illuminated a mere \SI{1}{\second} every \SI{120}{\second}. Since we did not observe any irregularities in our experiments known to be induced by light we conclude that our precautions were sufficient.

	\section{Graph Extraction}

		Given the obtained raw data, we discard all series that do not show proper network formation. Thus the number of usable datasets is reduced to $54$. For the remaining series we seek to describe the characteristic \P networks by equivalent graphs. In addition to capturing the topology of the networks, we want to obtain a precise measure on the length and width of each vein observed. Furthermore, we want to establish the positions of the junctions of the veins in the image. Thus, we want to compute a weighted graph, whose nodes carry the positions of the junctions in the plane and whose edges carry weights corresponding to the length and the width of the observed veins. 

		To compute such a graph representation we rely on \NEFI. This tool takes as input an image from the raw dataset depicting a network and returns a faithful representation of this network in form of a weighted undirected graph. \NEFI offers several different algorithms and a variety of settings to do graph extraction. Some experimentation was necessary to find a sequence of algorithms, a so-called pipeline, such that the returned graph representations preserve as much information as possible. The pipeline has been stored and is part of the dataset for reasons of reproducibility. Asserting the effectiveness of a pipeline is convenient and easy, since the tool allows to visually compare the computed graph with the network in the input image by drawing the former ontop of the latter. An example can be seen in \Fref{fig:sequence_5}. For a more detailed discussion of the reliability of \NEFI and how to use it, we refer to its \href{http://nefi.mpi-inf.mpg.de}{project page} and companion paper~\cite{dirnberger2015nefi}.

		The main caveat of \NEFI is that, like any form of image processing or computer vision, the quality of the output strongly depends on the quality of the input. To obtain good results with this tool, the input image must be of high contrast and void of strong color gradients and other detrimental effects~\cite{dirnberger2015nefi}. Due to the design of our experiments these requirements are largely satisfied. However, due to its implementation \NEFI struggles with parts of the image that do not depict networks. In particular it fails to process regions depicting the inoculation line and the apical zone. For the network extraction to succeed these areas must be removed from the images or equivalently a region of interest must be defined excluding such areas. For consistency we define a specific region of interest for each given image series of the raw data set. A typical region of interest is seen in \Fref{fig:sequence_4}.

		To do so, we visually inspected every single image\footnote{The reader is absolutely right to assume that inspecting several thousand images required an extraordinary amount of patience and time.} of every sequence in order to decide on a maximal region of interest containing properly formed networks. It is common that somewhere within an image sequence \P starts to deviate from ``well-behaved'' growth, effectively disqualifying the sequence from this point on. Examples include \P suddenly reversing direction or spontaneously spawning new growing tips within an already established network. Thus we make two choices: First, for each sequence of images we find the longest usable subsequence and second, for each subsequence we decide on one region of interest. We store this information in small configuration files suitable for automated graph extraction. We point out that in general it has been beneficial if the choices of selection are made somewhat defensively, leading to a reduced likelihood of artifacts occurring in the graph detection process.  

		Given the configuration files and the extraction pipeline, \NEFI can be used to batch process sequences automatically. Note that for some series, partially containing strong color gradients in the background, \NEFI failed to properly segment the input images resulting in unusable graphs. This situation can be detected easily by inspection of the segmented images or the graph drawings produced by \NEFI. The affected series and the resulting graphs are then excluded from further processing. While this reduces the number of usable graph series to $36$, the remaining graphs capture the topology of the original \P networks exceptionally well. 

		Note that the raw graphs obtained so far are likely to contain artifacts such as isolated nodes and dead-ends, see \Fref{fig:sequence_5}. This is to be expected since \NEFI cannot reliably resolve structures that are very fine grained, \eg veins in the network with a width of less than $5$~pixel. As a result small structures in the graph break up into several disconnected parts. In a similar fashion spurious isolated nodes can enter the computed graph. We strongly recommend anyone considering to work with the raw graphs to carefully inspect them first in order to assess whether these artifacts need to be removed using filters. In our experience, a moderate amount of filtering is always appropriate and considerably improves quality of the graphs, \ie the degree to which they resemble the original \P networks.

		To deal with the mentioned artifacts \NEFI comes with the possibility to apply filters capable of removing isolated nodes and dead ends. We have filtered all raw graphs to obtain the final set of graphs which we store in several file formats. In particular we removed all edges that are not on a cycle, \ie dead ends are removed, and kept only the largest connected component. For all but the finest of veins the filtered graphs capture the structure of the original \P networks extremely well. Furthermore they carry precise information about node positions, edge widths and edge lengths. For a detailed description of how to work with the actual graph files produced by \NEFI we refer to its \href{http://nefi.mpi-inf.mpg.de}{project page}.

		Lastly, we point out that the process of graph extraction described here is geared towards answering a particular set of research questions, see~\cite{dirnberger2016}. For a different set of questions changes may be appropriate and necessary. They can easily be implemented by starting with the raw graphs and applying different filters. Also, it is not difficult to go back even further to the original image sequences and select different regions of interest and different subsequences, leading to different series of raw graphs. Given \NEFI and the possibility to use configuration files to automate the graph extraction, it becomes possible to adapt the data in the \data to various particular needs.

	\section{Continued Production of Sclerotia}\label{sec:sclerotia}

		Here we describe how to continuously produce dried sclerotia~\cite{lifecycle}, from a small initial sample\footnote{Starting samples of dried \P sclerotia can be purchased online or shared via the \P community, \eg at the~\href{http://slimoco.ning.com/}{Slime Mold Collective}.} of the same. First, we prepare a \SI{22 x 32}{\centi\metre} plastic dish with a layer of \SI{1}{\percent} Agar (Kobe I) and line up a generous amount of sclerotia evenly along the short side of the dish, forming an inoculation line. Ideally, the dish is kept in a light-proof box with high humidity at a temperature around \SI{22}{\degree}. That being said, forest-dwelling \P is rather forgiving and will make do with room-temperature and room-humidity provided it is properly shielded from light.

		After \SIrange{10}{14}{\hour} \P changes from the sclerotia state to the plasmodium stage and starts to explore the dish. From this point on we recommend to feed it with oat flakes every \SIrange{4}{6}{\hour}. To do so, distribute a small amount of oat flakes evenly across the inoculation line. Given the additional nutrients the organism will rapidly cover the whole dish at a rate of approximately \SI{1}{\centi\metre\per\hour}. Shortly before it reaches the far side of the dish we proceed to transfer the organism to new containers. The mass covering the oat-flakes is moved to a new dish with ample room to keep the growth of the plasmodium going in a continuous fashion. The remainder of the organism is transfered to a bucket allowing it to dry, thus triggering the transition back to the stage of sclerotia. A detailed description of both steps follows.

		First, we move the plasmodium at the inoculation line together with the accumulated overgrown oat-flakes to a new \SI{22 x 32}{\centi\metre} dish prepared with agar gel. To do so we cut the agar of the old dish underneath the inoculation line into suitable pieces, which we transfer to the new dish using a small spatula. We arrange the pieces of gel carrying the plasmodium such that a new inoculation line is formed. Soon the plasmodium will proceed to conquer the new dish. From time to time we add a small amount of oat flakes to the starting line to make sure \P keeps growing steadily. Transferring to a new dish serves to keep the growth of the organism going continuously. This step can be omitted if continued production of plasmodium is not desired. If at some point the growth of the slime mold seems to have come to an unintentional halt, we recommend to make sure it is not to dry by carefully moistening \P using distillate water and a plant sprayer.

		Next, we move the remaining contents of the overgrown gel to a \SI{40}{\centi\metre} high and \SI{25}{\centi\metre} wide ordinary cylindrical plastic bucket. In particular we make sure to transfer the whole growing front since it constitutes a significant part of the biomass of the organism. Before we transfer the organism, we cover the bottom of the bucket with tissue paper and stick moist filter paper to the walls of the bucket such that its inside is completely covered. After a while the plasmodium starts exploring the bucket and is naturally drawn towards the moistened areas. Eventually it will move away from the drier bottom and move along the bucket walls covering the filter paper. As soon as most of the filter paper is occupied by \P, the paper can be removed, wrapped up and stored to dry, triggering the transition from plasmodium to sclerotia. Letting the wrapped up filter paper dry in a cardboard box is simple and effective. After approximately $3$ days, the cell mass is fully dry and in the state of sclerotia. They can easily be handled and stored for later use. Once dried, the sclerotia can be scraped off the filter paper and used with precision in the main experiments.
