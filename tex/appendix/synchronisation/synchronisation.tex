%!TEX root = ../../../thesis.tex

\chapter{Code Listings}\label{app:code}

	Here we list the basic code snippets which constitute the heart of the numerical simulation of our model. They serve merely as an illustration and written documentation of the simulation setup and are not intended to provide a run-able program\footnote{The snippets listed here fully describe the entire non-trivial part of our simulation. We omit code of minor concern such as the creation of test graphs, initialization and data logging.}. A complete run-able version of our simulation is available for download at the \href{https://people.mpi-inf.mpg.de/~mtd/}{authors page}.

\begin{codesnippet}

\captionof{listing}[Model implementation - Required libraries]{Required python libraries.  Here the excellent \mintinline{python}{networkx} provides graph support~\cite{networkx}.}
\label{code:libaries}

\begin{minted}[linenos=true]{python}
from itertools import combinations
import networkx as nx
import numpy as np
\end{minted}

\end{codesnippet}


\begin{codesnippet}
\captionof{listing}[Model implementation - Core of the simulation]{Core of the simulation. Note that the sequence of updates is different from \Fref{sec:discrete}. This change simplifies computation but does not affect the results.}
\label{code:simulation}
\begin{minted}[linenos=true,linenos=true,tabsize=2,breaklines]{python}
def simulation(graph, epsilon = 0.01, max_iterations = 3000):

	number_of_iterations = 0

	while True:
		
		update_node_voltage(graph)
		update_voltage_source(graph, epsilon)
		update_edge_capacitor_voltage(graph, epsilon)
		
		number_of_iterations += 1	

		if number_of_iterations >= max_iterations:
			print "Reached the maximum number of iterations:", max_iterations
			break

	return graph
\end{minted}

\end{codesnippet}


\begin{codesnippet}
\captionof{listing}[Model implementation - Update node voltages]{Function updating the voltages for all nodes. Implements \Fref{eq:uv}. Using the dictionaries \mintinline{python}{R_map} and \mintinline{python}{Uc_map} in conjunction with \mintinline{python}{np.prod} and \mintinline{python}{combinations} yields a convenient way to compute the required sums and products.}
\label{code:node_voltage}
\begin{minted}[linenos=true,tabsize=2,breaklines]{python}

def update_node_voltage(graph):
	# d_u["U"] --> voltage at node u
	# d_e["Uc"] --> capacitor voltage at edge e
	# d_e["R?"] --> resistance at edge e
	# d_e["Up_?"] --> current controlled voltage source voltage

	for u, d_u in graph.nodes_iter(data=True):
			
		R_map = {}
		Uc_map = {}
		edge_counter = 0

		if graph.degree(u) == 1:

			for _,_,d_e in graph.in_edges_iter(u, data=True):
				d_u["U"] = d_e["Uc"] - d_e["Up_2"]

			for _,_,d_e in graph.out_edges_iter(u, data=True):
				d_u["U"] = d_e["Uc"] - d_e["Up_1"]

		elif graph.degree(u) >= 2:

			for _,_,d_e in graph.in_edges_iter(u, data=True):

				R_map[edge_counter] = d_e["R2"]
				Uc_map[edge_counter] = d_e["Uc"] - d_e["Up_2"]
				edge_counter += 1

			for _,_,d_e in graph.out_edges_iter(u, data=True):

				R_map[edge_counter] = d_e["R1"]
				Uc_map[edge_counter] = d_e["Uc"] - d_e["Up_1"]
				edge_counter += 1


			R_product = np.prod( R_map.values() )
			combined_resistance = sum( [ np.prod(e) for e in combinations(R_map.values(),len(R_map.values())-1) ])

			current = 0.0

			for key in R_map.keys():

				current += R_product * (float(Uc_map[key])/R_map[key])

			voltage = current / combined_resistance
		
			d_u["U"] = voltage

\end{minted}

\end{codesnippet}

\begin{codesnippet}
\captionof{listing}[Model implementation - Update capacitor voltages]{Function updating the capacitor voltages for all edges. The quantity \mintinline{python}{Ucs} is the sum over all capacitor voltages and invariant by \Fref{lem:inv_cont} since $C_e = C$ for all $e \in G$ as determined by the initial conditions.}
\label{code:capacitor_voltage}
\begin{minted}[linenos=true,tabsize=2,breaklines]{python}

def update_edge_capacitor_voltage(graph, epsilon):
	# d_e["I1"] --> current ip at edge e
	# d_e["I2"] --> current ip* at edge e
	# d_e["C"] --> capacity of capacitor at edge e
	# d_e["Up_old_?"] --> previous value of d_e["Up_?"]
	
	for n in graph.nodes_iter():

		# examine out edges for node n
		for u,v,d_e in graph.out_edges_iter(n, data=True):

			C = d_e["C"]
			r1 = d_e["R1"]
			r2 = d_e["R2"]
			p1 = d_e["Up_old_1"]
			p2 = d_e["Up_old_2"]
						
			Ux = graph.node[u]["U"]
			Uy = graph.node[v]["U"]

			a = euler_step_size/(r1*C)
			b = euler_step_size/(r2*C)
			
			Uc = Ux*a + Uy*b + d_e["Uc"] + euler_step_size/(r1*C)*((d_e["Up_old_1"] + d_e["Up_old_2"]) - 2*d_e["Uc"]) 

			d_e["Uc"] = Uc

	# the following quantity is an invariant
	Ucs = sum(nx.get_edge_attributes(graph,"Uc").values())		
\end{minted}


\end{codesnippet}


\begin{codesnippet}
\captionof{listing}[Model implementation - Update current controlled voltage sources]{Updates for current controlled voltage source. Implements \Fref{eq:left_up_18}, \Fref{eq:left_up_19} and \Fref{eq:left_up_20} for the l.h.s.\ and \Fref{eq:right_up_21}, \Fref{eq:right_up_22} and \Fref{eq:right_up_23} for r.h.s.\ voltage source respectively.}
\label{code:current_controlled_voltage_sources}
\begin{minted}[autogobble=true,tabsize=2,linenos=true,breaklines]{python}
def update_current_controlled_voltage_sources(graph, epsilon):
	# d_e["beta_?"] --> property beta
	# d_e["alpha_?"] --> property alpha
	# d_e["Upmax"] --> cap for current controlled voltage source
	
	def voltage_delay(Up_target, Up_current, alpha, epsilon):
		
		c = np.exp((-1)*epsilon*alpha)
		
		return Up_target * (1 - c) + Up_current * c

		for n, d_u in graph.nodes_iter(data=True):

			for u,v,d_e in graph.out_edges_iter(n,data=True):

				# equation 18, here "Uc" = Uc(r-1) in the text
				I1 = (d_u["U"] + d_e["Up_1"] - d_e["Uc"])/d_e["R1"]
				d_e["I1"] = I1

				beta_1 = d_e["beta_factor"]*d_e["R1"]
				Up_max = d_e["Upmax"]
				Up_min = (-1)*d_e["Upmax"]
				alpha = d_e["alpha_factor"]/(d_e["R1"]*d_e["C"])

				# equation 19
				Up_target = max(min(Up_max, beta_1 * I1), Up_min)
	
				Up_current = voltage_delay(Up_target, Up_current = d_e["Up_1"], alpha, epsilon)

				d_e["Up_old_1"] = d_e["Up_1"]
				d_e["Up_1"] = Up_current

			for u,v,d_e in graph.in_edges_iter(n,data=True):

				# equation 21, here "Uc" = Uc(r-1) in the text
				I2 = (d_u["U"] + d_e["Up_2"] - d_e["Uc"])/d_e["R2"]
				d_e["I2"] = I2

				beta_2 = d_e["beta_factor"]*d_e["R2"]
				Up_max = d_e["Upmax"]
				Up_min = (-1)*d_e["Upmax"]
				alpha = d_e["alpha_factor"]/(d_e["R2"]*d_e["C"])

				# equation 22
				Up_target = max(min(Up_max, beta_2 * I2), Up_min)
				# equation 23
				Up_current = voltage_delay(Up_target, Up_current = d_e["Up_2"], alpha, epsilon)

				d_e["Up_old_2"] = d_e["Up_2"]
				d_e["Up_2"] = Up_current
\end{minted}


\end{codesnippet}


