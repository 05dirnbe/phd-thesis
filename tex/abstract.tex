%!TEX root = ../thesis.tex
\null\vfill

\section*{Abstract}
	
	This doctoral thesis works towards distributed natural computing inspired by the slime mold \Pp. The vein networks formed by this organism presumably support efficient transport of protoplasmic fluid. Devising models that capture the natural efficiency of this organism and form a suitable basis for the development of natural computing algorithms is an interesting and challenging goal.
	
	We start working towards this goal by designing and executing wet-lab experiments geared towards producing a large number of images of the vein networks of \P. Next, we turn the depicted vein networks into graphs using our own custom software called \NEFI. This enables a detailed numerical study, yielding catalog of characterizing observables spanning a wide array of different graph properties. To share our results and data, \ie raw experimental data, graphs and analysis results, we introduce a dedicated repository revolving around slime mold data, the \SMGR. The purpose of this repository is to promote data reuse and to foster a practice of increased data sharing. 

	Finally we present a model based on interacting electronic circuits including current controlled voltage sources, which mimics the emergent flow patterns observed in live \P. The model is simple, distributed and robust to changes in the underlying network topology. Thus it constitutes a promising basis for the development of distributed natural computing algorithms.

\vfill

\section*{Zusammenfassung}

	Diese Dissertation dient als Vorarbeit f\"ur den Entwurf von verteilten Algorithmen, inspiriert durch den Schleimpilz \Pp. Es wird vermutet, dass die Venen-Netze dieses Organismus den effizienten Transport von protoplasmischer Fl\"uessigkeit erm\"oglichen. Die Herleitung von Modellen, welche sowohl die nat\"urliche Effizienz des Organismus wiederspiegeln, als auch eine geeignete Basis f\"ur den Entwurf von Algorithmen bieten, gilt weiterhin als schwierig.

	Wir n\"ahern uns diesem Ziel mittels Laborversuchen zur Produktion von zahlreichen Abbildungen von Venen-Netzwerken. Weiters f\"uhren wir die abgebildeten Netze in Graphen \"uber. Hierf\"ur verwenden wir usere eigene Software, genannt \NEFI. Diese erm\"oglicht eine numerische Studie der Graphen, welche einen Katalog von charakteristischen Grapheigenschaften liefert. Um die gewonnenen Erkenntnisse und Daten zu teilen, f\"uhren wir ein spezialisiertes Daten-Repository ein, genannt \SMGR. Hiermit beg\"unstigen wir die Wiederverwendung von Daten und f\"ordern das Teilen derselben.

	Abschliessend pr\"asentieren wir ein Modell, basierend auf elektrischen Elementen, einschlie{\ss}lich stromabh\"angigen Spannungsquellen, welches die Fl\"usse von \P nachahmt. Das Modell ist simple, verteilt und robust gegen\"uber topologischen \"Anderungen. Aus diesen Gr\"unden stellt es eine vielversprechende Basis f\"uer den Entwurf von verteilten Algorithmen dar.


\vfill
