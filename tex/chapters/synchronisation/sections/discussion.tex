%!TEX root = ../../../../thesis.tex
\section{Discussion}

	We have proposed a preliminary model representing the oscillatory flows observed in networks formed by \P as currents in electrical networks. Our approach is based on the three-element Windkessel model. This approach has successfully been used in the past to model human cardiovascular networks which strongly resemble the vein networks of \P. The major difference between the two is the fact that slime molds do not have one powerful central pump, \ie a heart, capable of producing flows. Rather, each and every vein of the slime mold network may act freely and contribute to the flow as an independent peristaltic pump. Since a vein may be connected to several neighboring veins, non-trivial local interactions arise. To model emergent global oscillations we propose an extension to the Windkessel model, namely current controlled voltage sources. The resulting electronic \Pes are then connected according to a given topology to form a \Pn. The electric currents induced by such networks exhibit complex emergent flow patterns reminiscent of the flows observed in live \P.

	In a symbiotic fashion we combine analytic and numerical methods to explore the characteristics of the resulting model. We find first and foremost, that our model can be discretized and solved to yield current flows for \Pn of arbitrary topology. In practice we restrict our simulations to simple graph classes of limited size. They illustrate that the model is fully distributed as it requires no central control. It exhibits self-organization leading to coordinated flows and global anti-phase entrainment. In addition to that, we determine that the model is robust to changes of the underlying topology of \Pns. These are qualities attributed to live \P. Furthermore, we hope that our model also inherits some of the natural efficiency attributed to the behavior of \P. We are reduced to hope in this instance, because there is no way of asserting the statement in a systematic way.

	When we began thinking about meaningful extensions to the Windkessel model, we noticed that we were frequently faced with decisions about what assumptions to accept during the initial stages of modeling. First and foremost we decided to keep the model simple enough to be able to proof facts about it. This is in line with our desire to obtain a prospective candidate model that would serve as a basis for future attempts of natural computing. Thus we placed significant emphasis on implementing simplifying assumptions. Needless to say, these come at the cost of a less accurate description of live \P. While this is unfortunate, in the context of computing inspired by nature it is of minor concern. In summary we decided to differentiate the original general model such, that modeling accuracy is sacrificed for ease of analytical treatment.

	Naturally, a different approach is necessary if your goal is synthesis of nature by means of computing. Here one seeks to obtain a model providing modeling accuracy and ultimately, predictive power. To do so, trading manageable complexity for a higher degree of modeling accuracy is acceptable. This entails replacing some of the following simplifying assumptions with more meaningful alternatives in order to obtain a model that is more interesting from a biophysical point of view. Perhaps most influential of all is the assumption that the electric resistance is constant and the same for all edges in the network. Note that the electrical resistance of \Pes translates to hydraulic resistance of veins in live \P, a quantity which strongly depends on the width of the veins. In fact, we have established in \Fref{sec:analysis_results} that the distribution of widths in real \P is not constant but is more likely to follow a gamma or log-normal distribution. Clearly, the model could be made more meaningful by incorporating width distributions of \P graphs. Furthermore, the topologies of real \P graphs which are conveniently available in the \SMGR could directly be used to run the model on.

	Incorporating these improvements including the move from simple small graphs to more complex \P graphs with more degrees of freedom for the edges likely shuts down our hopes of obtaining analytical statements about what is going on. Note that such a model may yet be solvable given the right numerical tools. It could still form a valid basis for natural computing, however the analysis of obtained algorithms is expected to range somewhere between challenging and seemingly impossible.

	Note that augmenting the Windkessel model such that modeling accuracy is maximized does not automatically yield a model with predictive power. What remains is to determine a way to set biophysically meaningful parameters for the \Pes. Unfortunately, the authors have no suggestion as to how to resolve this problem at present. A closer collaboration with biologist and biophysicists seems necessary to tackle this question.

	The authors are convinced that both described approaches supported by an augmented Windkessel model are valid and should be explored further. Indeed a manuscript is being prepared documenting our own attempts of deriving a distributed natural computing algorithm based on this model. While it is not necessary for a natural computing algorithm to closely resemble its source of inspiration, it would be nice if a connection between the way such an algorithm works and the way \P operates could be established. This task will likely require insights obtained in the pursuit of models of higher biophysical relevance and further strengthen the interdisciplinary appeal of natural computing with \P.
