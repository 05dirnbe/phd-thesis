%!TEX root = ../../../../thesis.tex

\section{Discrete Model}\label{sec:discrete}

In the following we propose a discrete round model, with rounds $r \in \IN$.
Again, a \Pn is specified by a graph $G(V,E)$ whose edges are labeled as in the continuous time model.
For simplicity we assume that all parameters $R,C,\alpha,\beta$ and $\hat{U}$ are the same for all edges.

The discrete model is derived from the continuous model using forward Euler integration. Let $\varepsilon > 0$ denote the step-size, \ie the time between two discrete rounds. We refer the reader to \Fref{app:code} for a basic implementation of the crucial updates defined in this section.

\medskip

\noindent
{\em Intially at $r=0$:}
For each edge $e$ in $G$ do:  
\begin{enumerate}
\item Set $u_{C,e}(r=0)$ and $u_{p,e}(r=0)$ as specified by the edge label.
\end{enumerate}

\medskip

\noindent
{\em At each round $r \ge 1$:}
\begin{enumerate}
\item Update node voltages (\Fref{code:node_voltage}): in the continuous model we determine node voltages according to \Fref{eq:uv}.
We follow the same approach in the discrete model.
Let $v$ be a node in $G$. Without loss of generality we assume that all edges incident to $v$ are oriented as in \Fref{fig:vein} with $v$ connected to $x$.
Then we have
\begin{align}
u_v(r-1) = \frac{1}{|E(v)|}\sum_{e \in E(v)}(u_{C,e}(r-1)-u_{p,e}(r-1))\,.\label{eq:dis:u}
\end{align}

\item Update capacitor voltages (\Fref{code:capacitor_voltage}): in the continuous model, for each edge $e=(v,w)$ in $G$, we have
\begin{align}
  C\frac{d u_{C,e}(t)}{dt} &= i_{C,e}(t) = i_{p,e}(t) + i'_{p,e}(t)\notag\\
  &= \frac{u_v(t)+u_w(t)+u_{p,e}(t)+u'_{p,e}(t)-2u_{C,e}(t)}{R}\,.
\end{align}
By first order discretization, we thus obtain
\begin{align}
  &u_{C,e}(r) = u_{C,e}(r-1) +\notag\\
  &+\frac{u_v(r-1)+u_w(r-1)+u_{p,e}(r-1)+u'_{p,e}(r-1)-2u_{C,e}(r-1)}{RC\varepsilon^{-1}}\,.\label{eq:dis:uC}
\end{align}

\item Update current controlled voltages (\Fref{code:current_controlled_voltage_sources}): for each edge $e$ in $G$:
\begin{align}
i_{p,e}(r-1) &= \frac{u_v(r-1)+u_{p,e}(r-1)-u_{C,e}(r-1)}{R}\,,\label{eq:left_up_18}\\  
u^*_{p,e}(r) &= \max(\min(\beta i_{p,e}(r-1),\hat{U}),-\hat{U})\,,\label{eq:left_up_19}\\
u_{p,e}(r)   &= u^*_{p,e}(r)(1-e^{-\alpha \varepsilon})+ u_{p,e}(r-1)e^{-\alpha \varepsilon}\,.\label{eq:left_up_20}
\end{align}
Analogously, for the right voltage source, set
\begin{align}
i'_{p,e}(r-1) &= \frac{u_w(r-1)+u_{p,e}(r-1)-u_{C,e}(r-1)}{R}\,,\label{eq:right_up_21}\\  
{u'}^*_{p,e}(r) &= \max(\min(\beta i'_{p,e}(r-1),\hat{U}),-\hat{U})\,,\label{eq:right_up_22}\\
u'_{p,e}(r)   &= {u'}^*_{p,e}(r)(1-e^{-\alpha \varepsilon})+ u'_{p,e}(r-1)e^{-\alpha \varepsilon}\,.\label{eq:right_up_23}
\end{align}

\end{enumerate}

\medskip


\section{Basic Properties of Discrete Model}\label{sec:basic_disc}

% We show that the dynamic system of capacitor voltages can be written as a linear system of potentially changing matrices.

% \begin{lem}\label{lem:stochastic}
% Let $G$ be a physarum system and $u_C(r)$ be the column-vector formed by all $u_{C,e}$, where $e$ is an edge in $G$, in an arbitrary fixed order.
% If $\frac{2\varepsilon}{RC} < 1$ then,  
% \begin{align}
% \forall r\ge 1:\,u_C(r)=A(t) \cdot u_C(r-1)\,,
% \end{align}
% with $A(t)$ being a matrix whose column-sums are $1$, diagonal entries lower bounded by some $\gamma > 0$, and $A_{e,e'} = 0$
%   if $e$ and $e'$ are non-incident.
% \end{lem}
% \begin{proof}
% Let $r > 0$.
% From \eqref{eq:dis:uC} we have
% \begin{align}
%   &u_{C,e}(r) = \left(1-\frac{2\varepsilon}{RC}\right)u_{C,e}(r-1) +\notag\\
%   &+\frac{u_v(r-1)+u_{p,e}(r-1)}{RC\varepsilon^{-1}} + \frac{u_w(r-1)+u'_{p,e}(r-1)}{RC\varepsilon^{-1}}\,.\label{eq:dis:uC2}
% \end{align}
% For an edge $e = (v,w)$, we denote $v(e) = v$ and $w(e) = w$.
% From \eqref{eq:dis:u}, we obtain
% \begin{align}
% \sum_{e \in E}u_{v(e)}(r-1) = \frac{1}{|E(v)|}\sum_{e \in E}\sum_{e' \in E(v(e))}(u_{C,e'}(r-1)-u_{p,e'}(r-1))\,,\label{eq:dis:u:v1}\\
% \sum_{e \in E}u_{w(e)}(r-1) = \frac{1}{|E(w)|}\sum_{e \in E}\sum_{e' \in E(w(e))}(u_{C,e'}(r-1)-u_{p,e'}(r-1))\,.\label{eq:dis:u:w1}
% \end{align}
% ....

% \end{proof}

% From Lemma~\ref{lem:stochastic} we obtain the following analogon to Lemma~\ref{lem:inv_cont}.
% \begin{lem}\label{cor:inv_disc}
% For all physarum networks, there exists a $U \in \IR$ such that,
% \begin{align}
% \forall r\ge 0:\,\sum_{e \in E}u_{C,e}(r) = U\,.
% \end{align}
% \end{lem}


In the following we establish the discretized version of Lemma~\ref{lem:inv_cont}. This proof is important since it guarantees that the discretization we introduced in the previous section to solve the model does not break anything numerically. In particular, we know that the sum of the charges stored in the capacitors is still conserved.

\begin{lem}\label{cor:inv_disc}
For all Physarum networks, there exists a $Q \in \IR$ such that
\begin{align}
\forall r \ge 0:\,\sum_{e \in E}C_e u_{C,e}(r) = Q\,.
\end{align}
\end{lem}

\begin{proof}
Let $r > 0$. By rewriting \Fref{eq:dis:uC} we have

\begin{align}
&u_{C,e}(r) = \left( 1 - \frac{2 \epsilon}{RC}\right) u_{C,e}(r - 1)  \\\notag
&+\epsilon \frac{u_v(r - 1) + u_w(r-1)}{RC} + \epsilon \frac{u_{p,e}(r-1) + u'{p,e}(r - 1)}{RC}\,.
\end{align}
We may sum over all edges $e \in E$ on both sides to obtain

\begin{align}\label{eq:key}
&\sum_{e\in E} u_{C,e}(r) = \left( 1 - \frac{2 \epsilon}{RC}\right) \sum_{e\in E} u_{C,e}(r - 1)  \\\notag
&+ \frac{\epsilon}{RC}\left( \sum_{e\in E} u_{v(e)}(r - 1) + \sum_{e\in E} u_{w(e)}(r-1) \right) \\\notag
&+ \frac{\epsilon}{RC} \sum_{e\in E} \left(u_{p,e}(r-1) + u'_{p,e}(r - 1)\right)\,.
\end{align}
Here $v(e)$ denotes the tail of edge $e$ and $w(e)$ the head. We apply this distinction when we sum over all edges $e \in E$ in \Fref{eq:dis:u}, to obtain
\begin{align}
&\sum_{e \in E} u_{v(e)}(r-1) = \sum_{e \in E}\frac{1}{|E(v(e))|}\sum_{e' \in E(v(e))}(u_{C,e'}(r-1)-u_{p,e'}(r-1))\,,\label{eq:dis:u:v1}\\
&\sum_{e \in E} u_{w(e)}(r-1) = \sum_{e \in E}\frac{1}{|E(w(e))|}\sum_{e' \in E(w(e))}(u_{C,e'}(r-1)-u_{p,e'}'(r-1))\,.\label{eq:dis:u:w1}
\end{align}
Where we assume that the edges $E(v(e))$ are out-edges while the $E(w(e))$ are in-edges. This assumption is convenient for the purpose of analysis. however, it has no further implications since \Pes are symmetric. 

Next we substitute \eqref{eq:dis:u:v1} and \eqref{eq:dis:u:w1} into \Fref{eq:key} and get
\begin{align}\label{eq:key_substituted}
&\sum_{e\in E} u_{C,e}(r) = \left( 1 - \frac{2 \epsilon}{RC}\right) \sum_{e\in E} u_{C,e}(r - 1)  \\\notag
&+ \frac{\epsilon}{RC}\left( \sum_{e \in E}\frac{1}{|E(v(e))|}\sum_{e' \in E(v(e))}(u_{C,e'}(r-1)-u_{p,e'}(r-1)) \right)\\\notag
&+ \frac{\epsilon}{RC}\left( \sum_{e \in E}\frac{1}{|E(w(e))|}\sum_{e' \in E(w(e))}(u_{C,e'}(r-1)-u_{p,e'}'(r-1)) \right) \\\notag
&+ \frac{\epsilon}{RC} \sum_{e\in E} \left(u_{p,e}(r-1) + u'_{p,e}(r - 1)\right)\,.
\end{align}
Seeking to simplify the double summations appearing in \Fref{eq:key_substituted}, we first point out that for the sums regarding the $u_{C,e'}(r - 1)$ we have

\begin{align}\label{eq:simpler_1}
& \sum_{e\in E} \frac{1}{|E(v(e))|} \sum_{e' \in E(v(e))} u_{C,e'}(r - 1) = \sum_{e' \in E} u_{C,e'}(r-1) \sum_{e \in E; v(e) = v(e')}  \frac{1}{|E(v(e))|} \\\notag
&= \sum_{e' \in E} u_{C,e'}(r-1) = \sum_{e \in E} u_{C,e}(r-1)\,,
\end{align}
where the summations have been interchanged to get rid of one of the sums. Similarly we examine the sums regarding the $u_{p,e'}(r-1)$ and find

\begin{align}\label{eq:simpler_2}
& \sum_{e\in E} \frac{1}{|E(v(e))|} \sum_{e' \in E(v(e))} u_{p,e'}(r - 1) = \sum_{e' \in E} u_{p,e'}(r-1) = \sum_{e \in E} u_{p,e}(r-1)\,,
\end{align}
and analogously for the sums regarding the $u'_{p,e'}(r-1)$ it reads
\begin{align}\label{eq:simpler_3}
& \sum_{e\in E} \frac{1}{|E(w(e))|} \sum_{e' \in E(w(e))} u'_{p,e'}(r - 1) = \sum_{e' \in E} u'_{p,e'}(r-1) = \sum_{e \in E} u'_{p,e'}(r-1)\,.
\end{align}
Finally, we substitute \eqref{eq:simpler_1}, \eqref{eq:simpler_2} and \eqref{eq:simpler_3} into \Fref{eq:key_substituted} to arrive at

\begin{align}\label{eq:key_substituted_2}
&\sum_{e\in E} u_{C,e}(r) = \left( 1 - \frac{2 \epsilon}{RC}\right) \sum_{e\in E} u_{C,e}(r - 1)  \\\notag
&+ \frac{\epsilon}{RC}\left( \sum_{e \in E} u_{C,e}(r-1) - \sum_{e \in E} u_{p,e}(r-1) + \sum_{e \in E} u_{C,e}(r-1) -\sum_{e \in E} u'_{p,e'}(r-1) \right)\\\notag
&+ \frac{\epsilon}{RC} \sum_{e\in E} \left(u_{p,e}(r-1) + u'_{p,e}(r - 1)\right) \\\notag
&= \sum_{e\in E} u_{C,e}(r - 1)\,.
\end{align}
Thus we have 
\begin{align}
	\sum_e u_{C,e}(r) = \sum_e u_{C,e}(r - 1) =  \sum_e u_{C,e}(r - 2) = \ldots = \sum_e u_{C,e}(0) = const.\,.
\end{align}
From the fact that that all $C_e$ are constant and equal the lemma follows immediately.

\end{proof}