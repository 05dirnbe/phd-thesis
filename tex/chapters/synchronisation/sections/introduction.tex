%!TEX root = ../../../../thesis.tex

\section{Introduction}

  Slime molds are interesting and complex organisms providing a rich substrate for interdisciplinary research. In the recent past, one member of the family, called \Pp~\cite{howard1931life}, has been discovered as a suitable medium for natural computing. Its outstanding morphological dynamics have been repeatedly connected to optimization processes capable of solving complex problems. Prominent examples include computing shortest paths~\cite{nakagaki2000intelligence,tero2006physarum,bonifaci2012physarum}, designing transport networks~\cite{tero2010rules,nakagaki2007minimum} or controlling robots~\cite{tsuda2004robust} amongst others. 

  One striking feature of \P is its ability to form and maintain a massive cell body in the form of a dynamic complex network of veins. These networks are highly adaptive and may change drastically in response to changing environmental conditions. This extraordinary functional plasticity allows \P to navigate its environment successfully in search for food. Efforts to improve our understanding of formation, structure and function of these networks are manifold and ongoing~\cite{Marwan419,tero2010rules,alim2013random,baumgarten2010plasmodial,baumgarten2013functional}.

  Similar to the way a typical mammalian vascular network ensures the circulation of blood, the veins of the plasmodium allow protoplasmic fluid to freely flow~\cite{kamiya1958studies}. The resulting fluid circulation is vital to the organism as a whole since it ensures that nutrients, nuclei and other relevant factors are equally available across an entire individual. Note this circulation is maintained naturally despite the ever changing and adapting underlying network of veins. 

  The protoplasmic flow itself is driven by quasi-periodic cross-sectional contractions that occur with a period of approximately \SI{100}{\second}~\cite{stewart1959protoplasmic,Wohlfarth-Bottermann15}. They are generated by a mesh consisting of actin and myosin fibers forming the walls of the tubular veins. Contractions cause a peristaltic pumping effect leading to net fluid transport, the so-called shuttle streaming~\cite{kamiya1959motive}. Resulting peristaltic pressure waves can be observed across the entire network inducing complex flow patterns~\cite{Nakagaki2000195}. These include periodic flow arrests and reversals in single veins on a time scale of \SI{50 \pm 5}{\second}. 

  It is believed that efficient and robust transport of protoplasm across the entire organism emerges from the interplay of dynamic network topology, quasi-periodic local contractions and complex flow patterns~\cite{alim2013random,teplov1991continuum}. Unfortunately, the details of such a process remain in the dark.

  From a biological point of view it can be argued that evolution has optimized the behavior of \P at least to the point where fluid transport is efficient and robust enough to survive. Another evolutionary advantage can be seen in the self-organizing character of the slime mold since the evolved neural circuitry, vital to the survival of more complex organism, is simply not necessary to \P. Indeed, \P lacks any type central control capable of coordinating its apparently coordinated behavior. It is intriguing to ask how this organism manages to organize efficient and robust fluid transport in a fully decentralized manner.

  From a computing point of view such properties are non-trivial and highly desirable. What \P seems to produce and maintain naturally is an (approximate) solution to the problem of distributing resources in a dynamically changing planar graph. This is an interesting and complex transport problem with various conceivable practical applications, particularly in the domain of operations research. For an overview of similar transport problems and their computational complexity see~\cite{Ausiello:1999:CAC:554706,4567876,Hillier:1986:IOR:27036}.

  As a result, modeling the behavior of the slime mold with the goal of deriving algorithm for transport problems is an interesting proposition. Provided such an algorithm can be obtained, it should have the following properties:

  \begin{itemize}
  \item The algorithm maintains a dynamic circulation of flow including flow reversals mimicking the flows observed in live \P. Since resources are transported with the flow, they go wherever the flow reaches.
  \item The algorithm is robust against changes in topology. Neither natural changes of network topology nor accidental disconnection of veins renders \P in a state from which it cannot recover. The algorithm should share this quality.
  \item The algorithm is distributed and requires no central control. As a result, complex global coordination of any sort must emerge from local interactions.
  \item The algorithm has a degree of efficiency. Based on the assumption that a certain degree of efficiency is necessary for \P to survive, one may hope that models and algorithms mimicking the organism, inherit this efficiency at least to some extent.
  \end{itemize}

  In this manuscript we present a model of \P intended to support the development of distributed natural computing algorithms. It is inspired by the modeling of the human cardiovascular system and yields emergent flow patterns similar to the ones displayed by the organism. This includes flow reversals and anti-phase oscillations. Our model aspires to the properties listed above and aims at replicating the way \P distributes resources across its vein network. In particular it maps the oscillatory behavior of \P to simple interacting electronic circuits designed to mimic the effect of peristaltic pumping. The model is fully distributed and exhibits emergent dynamic flow patterns. Due to its relative simplicity it remains amenable to analytic treatment. \emph{In silico} investigations are presented in support of the validity of our approach which demonstrate the behavior of our model.

  The presented model may serve as a basis for future investigations in the context of natural computing. Two primary directions discussed in \Fref{sec:natural_computing} are possible: a) computing inspired by nature and b) the synthesis of nature by means of computing. For the former, one may utilize the model in an effort to develop novel distributed natural computing algorithms inspired by \P. The latter may aim towards refining the model with the goal of improving both its accuracy and predictive power. Both approaches present distinct challenges and both approaches deserve future exploration.
