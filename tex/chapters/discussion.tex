%!TEX root = ../../thesis.tex
\chapter{Summary}

	
	The present thesis intends to pave the way towards distributed natural computing inspired by the slime mold \Pp. The networks formed by this humble organism exhibit an extraordinary plasticity and presumably support efficient circulation of protoplasmic fluid through them. For this reason the complex dynamics observed in \P have repeatedly been compared to optimization processes. 
	Devising models that both capture the natural efficiency of this organism and at the same time form a suitable basis for the development of natural computing algorithms, is an interesting an challenging proposition.
	
	This thesis documents a series of sequential research efforts that build on-top of each other, working towards this goal. These can be subdivided into three broad categories: Experimental work, data analysis and modeling efforts.

	In the first category we design and execute a number of wet-lab experiments with \P. The experimental setup is geared towards producing images which document the topology of \P vein networks and their time development. Next, we turn the images depicting networks into actual graphs enabling subsequent analysis. To do this we develop a custom software called \NEFI. At this point, we make all obtained data, \ie raw images and corresponding graphs available to everyone. To facilitate this we introduce a dedicated publicly available repository revolving around slime mold data, the \SMGR. The goal of this repository is to enable data reuse and to invite others to join in our efforts of fostering a practice of increased data sharing.

	In the data analysis category of this thesis we scrutinize the \P graphs we obtained previously. To this end we design a set of observables aimed at capturing various properties of the vein networks. These include structural information about paths, face cycles as well as graph cuts. Finally we investigate percolation properties gauging the robustness of networks formed by \P. The results obtained by us form an extensive catalog of characterizing data which help quantify properties of slime mold networks.

	The last category of this thesis is concerned with modeling of \P which is the immediate precursor to deriving natural computing algorithms. Here we rely on electronic elements to model the dynamic flows observed in live the live organism. Our approach is inspired by the modeling of the human cardiovascular system but introduces additional current controlled voltage sources to mimic the effects of peristaltic pumping. These elements are connected according to a given topology in order to form electronic networks representing vein networks of \P. The model is simple but has the advantage that it analytically tractable on top of being solvable numerically. A preliminary \emph{in silico} exploration of its properties shows, that just like \P, it operates fully decentralized and is robust against changes in the underlying network topology. Above all, it yields complex current flows that resemble flow patterns observed in live \P. In particular, emergent anti-phase entrainment and flow reversals are observed. As such the model we obtain constitutes a promising candidate for the development of distributed natural computing algorithms inspired by \P. Efforts in this direction are underway and will be reported separately from this thesis.

	What started as a fascination with existing natural computing algorithms inspired by \P, organically developed into a desire\footnote{Some would even use the term ``obsession''.} to work towards our own natural computing approach. Little did we know when we started that we were embarking on a rather interdisciplinary journey.

	Note that at every step of the way we put a strong emphasis on making sure that everything we do, is at least as useful to others as it is to us. Needless to say, this almost always required us to go the so-called ``extra mile'' investing considerable time and effort. However, the idea of creating additional value for others, on-top of obtaining novel scientific results, is rather satisfying and thus quickly became an underlying principle of our work. Today it can be seen resonating throughout all of this thesis. 

	Our network extraction software \NEFI for instance was designed to process images depicting networks originating from various domains. In particular, it is not limited to the applications presented in this thesis. Rather, our tool and its multi-purpose design has the potential to be of continued use in a variety of unforeseen applications.

	The same can be said about our slime mold data repository, the \SMGR. Anyone can download its data and start using it right away. At the same time experts are invited to contribute their hard earned data and results to the \SMGR in order to increase the reach and impact of their work. We hope that the idea of the \SMGR will resonate with the research community concerned with slime molds.

	Our efforts of characterizing the networks formed by \P yielded a catalog of observables spanning a wide array of different graph properties. While the obtained results are not exactly ground-breaking, they have potential implications for evaluating and guiding all sorts of theoretical modeling approaches regarding \P. Model predictions that agree with data in the catalog increases the trust in a given model. At the same time discrepancies between predictions and catalog data hopefully suggest improvements to the model. Thus, the data and results presented in this thesis may be beneficial to the modeling efforts of others.

	Last but not least, there are our own modeling efforts regarding the flows observed in \P. Here we are working towards computing inspired by nature. By making various simplifying assumptions in the process of abstracting the flow dynamics of the organism, we emphasis low model complexity at the cost of reduced biophysical modeling power. Our aim is to strike a balance between both aspects such that the model remains manageable yet still captures the complex flow patters displayed by \P to a sufficient degree. Since our interests lie in the domain of natural computing, this trade-off is acceptable because a model of manageable complexity is more likely to eventually lead to algorithms which can be analyzed analytically. 

	Now, one may rightfully ask, why this trade-off cannot lean the other way, leading to a more reliable description of \P with an emphasis on biophysical accuracy at the expense of increased complexity. In other words placing an emphasis on the synthesis of the behavior of \P by means of computing. The first steps towards this direction probably start with omitting or improving some of the simplifying assumptions we made during the modeling process. It would be intriguing to investigate to which extend variations of the model are capable of producing meaningful predictions with regards to the behavior of \P. At present we believe that such an approach could be of pronounced interest regarding questions of biology and biophysics. It certainly appears feasible and suggests itself for future in-depth explorations.

	Note that the two different trade-offs between modeling power and complexity we illustrated are geared towards different goals and seem to support disjoint types of research questions. At first glance, they appear to be in stark contrast to each other. However, we propose to adopt a different view of the matter entirely. In fact, we strongly believe that they are merely two of many approaches which can and should support each other in contributing different clues to the study of \P. Combining various approaches to the study of this humble organism creates additional value but requires an open mind with a strong willingness to work in an interdisciplinary manner. The author tried to live up to this ideal with this thesis.

