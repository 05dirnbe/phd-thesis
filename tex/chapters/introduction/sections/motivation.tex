%!TEX root = ../../../../thesis.tex
\section{Motivation for this thesis}

	After introducing the most important experimental facts about \P and discussing the natural computing strategies inspired by them, we are now in a position to give the motivation of this thesis. Recall for a moment that some of the most successful approaches focus on modeling the morphological changes observed in \P. The physarum solver for instance replicates the changes in vein thickness as a function of throughput while the multi-agent approach mimics the network topology in a dynamic manner. What these and other approaches have in common is that they fail to capture another highly distinguishing feature of the slime mold: the periodic reversals of the direction of protoplasmic flow through veins. This feature is highly interesting for several reasons. 

	From a biophysical point of view it can be assumed that flow reversals are a product of the organisms efforts to keep the protoplasmic fluid circulating throughout the whole organism by means of organized peristaltic pumping. An efficient circulation of protoplasm likely is necessary for \P to survive since it ensures that nutrients, nuclei, signaling molecules \etc are equally available across the entire slime mold. Note that such a circulation is naturally maintained despite the dynamically changing and growing underlying network of veins. Again we stress the fact that \P lacks any form of nervous system or brain. As a result, there is no central control responsible for coordinating a global circulation. This implies that the circulation of flow is organized in a fully decentralized manner.

	From a computing point of view such properties are non-trivial yet highly desirable. What \P is naturally doing is approximating or obtaining a solution to the problem of distributing resources in a dynamically changing graph. This is an interesting and complex problem with various conceivable practical applications. As a result, it is desirable to study and obtain algorithms for this problem which are inspired by the workings of \P. Ideally such an algorithm has the following properties:

	\begin{itemize}
		\item the algorithm maintains a dynamic circulation of flow including flow reversals. Since resources are transported with the flow, they are distributed everywhere.
		\item the algorithm is robust against changes in topology. Neither natural changes of network topology nor accidental disconnection of veins renders the process in a state from which it cannot recover.
		\item the algorithm is distributed and requires no central control. As a result, complex global coordination of any sort must emerge from local interactions.
		\item the algorithm has a degree of efficiency. This may be assumed given that \P was able to survive up until now.
	\end{itemize}

	Paving the way towards such an algorithm constitutes the main motivation of this thesis. In particular we seek to contribute to the field of natural computing with \P through experimental and theoretical alike.

	Ultimately, we aim to find an model which yields a circulating flow including flow reversals for individual edges as displayed by \P. Ideally, this model covers the properties listed above and can be used to derive an efficient algorithm which mimics the way \P distributes resources all across its network of veins by means of peristaltic pumping. 

	The past has shown that successful modeling of natural phenomena approaches tend to build upon a reliable body of experimental data and results concerning the same. This serves two purposes. First experimental results help build intuition which is crucial in the formulation of models. Second, experimental data can be used after models and/or algorithms have been derived to assess the degree of which they resemble the workings of the natural phenomenon in question. Thus, before engaging in attempts to model the flows observed in \P, we must look towards experimental work.

	In this context, the oscillator experiments introduced in section something are most promising. They suggests to explore an approach based on the behaviour of synchronized coupled oscillators. Experimentally it was shown that dynamic flows, including flow reversals, follow from changing pressure gradients induced by organized thickness oscillations in the plasmodium of \P. Here the experimenters stress in particular that network properties such as the lengths and the thickness of veins critically influence the observed emergent behavior. 

	Thus, to gain insights in the dynamics of the flow, one may a) look at the flows directly, or b) study the vein networks themselves. Experimentally it is possible to track the flow of protoplasm through single veins of the organism for a certain amount of time. However, obtaining this information for a large vein networks including thousands of veins is outside of our capabilities and thus infeasible. Alternatively, one may study the topology of the networks on a large scale, which is the approach we follow in this thesis. 

	Thus, we obtain a detailed numerical characterization of the networks formed by \P. To enable this we conduct a large number of tailored wet-lab experiments designed to produce images of \P networks in the wet-lab. To evaluate and study the experimental data, we develop a dedicated software capable of turning images from the lab depicting \P networks into actual graphs. After processing the graphs, they can be treated and analyzed automatically using dedicated computer methods. Finally, we want our experimental insights to serve others in the same way the serve us. Thus we set up a dedicate repository designed to further the reuse and exchange of \P related data with a focus on slime mold graphs. These steps and their results, we consider a preliminary to our own natural computing approach. All our work is conducted in such a way that methods and results are readily available to the  natural computing community and everyone who is interested. 


	Thus, we conduct a large number of tailored wet-lab experiments designed to produce images of \P networks. To evaluate and study the experimental data, we develop a software called Network Extraction From Images or \NEFI. It is designed to turn images depicting \P networks into equivalent graphs. After processing of these \P graphs, we obtain a detailed numerical characterization of their properties searching for clues supporting our modeling efforts. 

	Naturally, we want all our results, \ie experimental raw data, processed data as well as our numerical studies, to serve others the same way the serve us. Thus we set up a dedicate repository, the so-called Slime Mold Graph Repository or \SMGR, designed to further the reuse and exchange of \P related data with a focus on slime mold graphs. All results presented in this thesis are readily available to the natural computing community and everyone who is interested online at \href{http://smgr.mpi-inf.mpg.de}{http://smgr.mpi-inf.mpg.de} and \href{http://nefi.mpi-inf.mpg.de/}{http://nefi.mpi-inf.mpg.de/}.

	This thesis documents the road towards our own natural computing approach inspired by flow reversals observed in \Pp. It certainly is an ambitious journey with an uncertain destination. However, as long as the steps we take along the way are interesting and scientifically valuable contribution to the natural computing community, we are happy to take them, one at a time.

	The remainder of this thesis is structured as follows: First we introduce \NEFI and the problem of extracting graphs from images. Network extraction is vital precursor to network analysis and key to making the raw data from our wet-lab experiments tractable. Next we describe the actual experiments we conduct with \P. We discuss the setup and the results we obtain. Furthermore we introduce the \SMGR where all our data and results are collected and published. Next we focus on the actual analysis of the obtained graphs where we present a systematic study of several relevant observables. Finally, we present our efforts of modeling flow reversals and discuss potential natural algorithms that can be based on them. The thesis closes with a discussion of the results presented and offers concluding remarks and suggestions for further research.










% reliable experimental data that describes a certain phenomenon and acts as a guide to the modeling process
% depending on the phenomenon under consideration this can require dedicated experimental methods and data processing approaches.
% the modelling process itself, where the challenge is to look past the intricate bio-physical processes that are observed in \P and find a suitable level of abstraction. It must capture the phenomenon of interest well, yet do away with a large part of the complexity of the organism.
% lastly turning the model into an algorithm.
% this requires a deep understanding of the model itself. The formal analysis of such algorithms belongs to the domain of computer science.

% what we want: get a simple model that captures the way \P uses thickness oscillations to generate pressure differences which generate fluid flow. 

% Why: because we assume that distribution stuff within its whole body is a vital task that \P needs to handle in order to survive. Here the hope is that the method \P uses to solve this task is also efficient. Experimental evidence is given by Alim \etal. The hope is to extract a bio-inspired fully distributed and fault tolerant algorithm capable of distributing goods in a graph.

% why not use the phyarum solver? Because it cannot capture oszillations and the tubes are passive elements. We do take inspiration from the physarum solver because we chose to model \P as an electric system. Studying the properties of the solver helped us realize that they are equivalent.

% why not use jones? because a particle model does not automatically lend itself to a graph abstraction. Might be possible to put it on a graph, however, certainly not natural.

% so what are you going to do?

% * we know that \P needs to solve the problem of efficient nutrient circulation. Alim \etal show evidence that \P does well. Alim \etal model veins as active elements.
% * we look at the oscillator experiment and realize that coupled oscillators replicate the syncronisation behavior. Here we realize that the topology and edge weights determine the behaviour the oscillator
% * we realize that we need to know as much as possible about the topology and the edge weights of \P networks when there is no food around.
% * we do the experiments, we write the software, we get the graphs, we publish everything
% * we look at the modelling of other vein based circulation systems such as the cardio vascular system. Here we learn how to model veins that are actively
% * We decide to model physarum as a system of oscillating electronic elements. In this model the veins are the active part and the nodes have a passive role.

% what is your contribution?

% * supporting the development of natural computing with \P by providing experimental data in form of images and graphs. To this we add NEFI and the SMGR. We hope that by doing so we enable others.

% * working towards the goal of mimicking the way \P distributes fluid all across its body. Ideally, we obtain an effecient, distributed algorithm.
