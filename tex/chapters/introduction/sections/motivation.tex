%!TEX root = ../../../../thesis.tex
\section{Motivation for this thesis}

given the form of \P it is not suptrising that graph algorithms pop up.

developing models that are defined on a graph seems appropriate.

in order to gain inspiration to write down such models there is several options

study the biological processes that go on in \P

study the networks themselves

the first one is complicated and intricate

the second one is the focus of this thesis

we want to be able to have \P networks available as graphs

to do so we need to do the experiments

we need to develop dedicated software to get the graphs

then we may study the graphs to get a better understanding of their properties.

be able to use the gained information to write down better models

be able to compare existing models agains the information obtained

this is the first part of the motivation

the second part seeks to make everything we gain available to everyone

sounds basic, but is not true in reality because experimental data rarely gets published on this scale.

as a result data is only available to select specialist groups who manage to make what was discussed above happen.

we try to end this by turning our results into a publicly available knowledge base.



reliable experimental data that describes a certain phenomenon and acts as a guide to the modeling process
depending on the phenomenon under consideration this can require dedicated experimental methods and data processing approaches.
the modelling process itself, where the challenge is to look past the intricate bio-physical processes that are observed in \P and find a suitable level of abstraction. It must capture the phenomenon of interest well, yet do away with a large part of the complexity of the organism.
lastly turning the model into an algorithm.
this requires a deep understanding of the model itself. The formal analysis of such algorithms belongs to the domain of computer science.
