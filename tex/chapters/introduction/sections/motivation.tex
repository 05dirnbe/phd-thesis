%!TEX root = ../../../../thesis.tex
\section{Motivation for this thesis}

	After introducing the most important experimental facts about \P and discussing the natural computing strategies that followed, we are now in a position to give the motivation of this thesis. Recall for a moment that the most successful approaches focus on capturing the morphological changes observed in \P. The physarum solver for instance replicates the changing  diameter of veins while the multi-agent approach that was discussed aims at mimicking the network topology in a dynamic manner. What these and other approaches have in common is that they are not able to capture another highly distinguishing feature of the slime mold: the periodic reversal of the direction of protoplasmic flows through veins. This feature is highly interesting for several reasons. 

	From a biophysical point of view it can be assumed that flow reversals are a product of the organisms efforts to keep the protoplasmic fluid circulating throughout the whole organism by means of organized peristaltic pumping. An efficient circulation of protoplasm likely is necessary for \P to survive since it ensures that nutrients, nuclei, signaling molecules \etc are equally available across the entire slime mold. Note that such a circulation is naturally maintained despite the dynamically changing and growing underlying network of veins. Also,note the fact that \P lacks any form of nervous system or brain. As a result, there is no central control responsible for coordinating a global circulation. This implies that the circulation of flow is organized in a fully distributed manner.

	From a computing point of view these properties are highly interesting. Phrased otherwise, what \P is naturally doing constitutes a way of approximating or obtaining a solution to the problem of routing resources in a dynamically changing graph. This is an interesting and complex problem with various practical applications which has been studied by computer scientists in the past. Obtaining an algorithm for this problem inspired by \Pp is thus desirable. One may aim to find a new algorithm with the following properties:

	\begin{itemize}
		\item the algorithm maintains a dynamic circulation of flow including flow reversals. Since resources are transported with the flow, they are distributed everywhere.
		\item the algorithm is robust against changes in topology. Neither natural changes of network topology nor accidental disconnection of veins renders the process in a state from which it cannot recover.
		\item the algorithm is distributed and requires no central control. As a result, complex global coordination of any sort must emerge from local interactions.
		\item the algorithm is efficient, which can be assumed given that \P was able to survive up until now.
	\end{itemize}

	Working towards such an algorithm inspired by \P, which covers as many of these properties as possible, constitutes the motivation of this thesis. Our approach implies a sequence of preliminary steps. 

	First and foremost, we aim to find an model which yields a circulating flow including flow reversals for individual edges as displayed by \P. Ideally, this model covers the properties listed above and can be used to derive an efficient algorithm which mimics the way \P distributes resources all across its network of veins by means of peristaltic pumping. 

	The past has shown that successful modeling of natural phenomena approaches tend to build upon a reliable body of experimental data and results concerning the same. This serves two purposes. First experimental results help build intuition which is crucial in the formulation of models. Second, experimental data can be used after models and/or algorithms have been derived to asses the degree of which they resemble the workings of the natural phenomenon in question. Thus, before engaging in attempts to model the flows observed in \P, we must look towards experimental work.

	In this context, the oscillator experiments introduced in section something are most promising. They suggests to explore an approach based on the behaviour of synchronized coupled oscillators. Experimentally it was shown that dynamic flows, including flow reversals, follow from changing pressure gradients induced by organized thickness oscillations in the plasmodium of \P. Here the experimenters stress in particular that network properties such as the lengths and the thickness of veins critically influence the observed emergent behavior. 

	Thus, to gain insights in the dynamics of the flow, one may a) look at the flows directly, or b) study the vein networks themselves. Experimentally it is possible to track the flow of protoplasm through single veins of the organism for a certain amount of time. However, obtaining this information for a large vein networks including thousands of veins is infeasible. An alternatively one may study the networks on a large scale, which is the approach we follow in this thesis. 

	Thus, we conduct a detailed numerical characterization of networks formed by \P. Such a characterization requires tailored experiments designed to produce images of \P networks in the wet-lab. Furthermore, a dedicated methods are needed, capable of turning images depicting networks into actual graphs which can be treated and analyzed in computer programs. Finally, we want our experimental insights to serve others in the same way the serve us. Thus we set up a dedicate repository designed to further the reuse and exchange of \P related data with a focus on slime mold graphs. These steps and their results, we consider a preliminary to our own natural computing approach. All our work is conducted in such a way that methods and results are readily available to the  natural computing community and everyone who is interested. 

	This thesis documents the road towards our own novel natural computing approach inspired by \Pp. It certainly is an ambitious journey with an uncertain destination. However, as long as the steps we take along the way are interesting and scientifically valuable contribution to the natural computing community, we are happy to take them, one at a time.










% reliable experimental data that describes a certain phenomenon and acts as a guide to the modeling process
% depending on the phenomenon under consideration this can require dedicated experimental methods and data processing approaches.
% the modelling process itself, where the challenge is to look past the intricate bio-physical processes that are observed in \P and find a suitable level of abstraction. It must capture the phenomenon of interest well, yet do away with a large part of the complexity of the organism.
% lastly turning the model into an algorithm.
% this requires a deep understanding of the model itself. The formal analysis of such algorithms belongs to the domain of computer science.

% what we want: get a simple model that captures the way \P uses thickness oscillations to generate pressure differences which generate fluid flow. 

% Why: because we assume that distribution stuff within its whole body is a vital task that \P needs to handle in order to survive. Here the hope is that the method \P uses to solve this task is also efficient. Experimental evidence is given by Alim \etal. The hope is to extract a bio-inspired fully distributed and fault tolerant algorithm capable of distributing goods in a graph.

% why not use the phyarum solver? Because it cannot capture oszillations and the tubes are passive elements. We do take inspiration from the physarum solver because we chose to model \P as an electric system. Studying the properties of the solver helped us realize that they are equivalent.

% why not use jones? because a particle model does not automatically lend itself to a graph abstraction. Might be possible to put it on a graph, however, certainly not natural.

% so what are you going to do?

% * we know that \P needs to solve the problem of efficient nutrient circulation. Alim \etal show evidence that \P does well. Alim \etal model veins as active elements.
% * we look at the oscillator experiment and realize that coupled oscillators replicate the syncronisation behavior. Here we realize that the topology and edge weights determine the behaviour the oscillator
% * we realize that we need to know as much as possible about the topology and the edge weights of \P networks when there is no food around.
% * we do the experiments, we write the software, we get the graphs, we publish everything
% * we look at the modelling of other vein based circulation systems such as the cardio vascular system. Here we learn how to model veins that are actively
% * We decide to model physarum as a system of oscillating electronic elements. In this model the veins are the active part and the nodes have a passive role.

% what is your contribution?

% * supporting the development of natural computing with \P by providing experimental data in form of images and graphs. To this we add NEFI and the SMGR. We hope that by doing so we enable others.

% * working towards the goal of mimicking the way \P distributes fluid all across its body. Ideally, we obtain an effecient, distributed algorithm.
