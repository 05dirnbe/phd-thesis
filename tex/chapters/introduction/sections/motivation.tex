%!TEX root = ../../../../thesis.tex
\section{Motivation and Outline}

	After surveying the most important experimental facts about \P and discussing the natural computing strategies inspired by them, we are now in a position to give the motivation of this thesis. 

	Recall for a moment that some of the most successful natural computing approaches are based on modeling the morphological changes observed in \P. The Physarum solver for instance replicates the changes in vein thickness as a function of throughput while the multi-agent approach mimics the network topology in a dynamic manner. What these and other approaches have in common is that they fail to capture another distinguishing feature of the slime mold: the periodic reversals of the direction of protoplasmic flow through veins. This feature is highly interesting for several reasons. 

	From a biophysical point of view it may be assumed that flow reversals are a byproduct of the organisms efforts to keep protoplasmic fluid circulating. An efficient circulation of protoplasm is beneficial, possibly necessary, for \P to survive since it ensures that nutrients, nuclei and other relevant factors are equally available across the entire individual. Note that such a circulation is naturally maintained despite the dynamically changing and growing underlying network of veins. Again we stress the fact that \P lacks any form of nervous system or brain. As a result, there is no central control to be made responsible for coordinating the apparently coordinated observed behavior. It is intriguing to ask how this organism manages to organize efficient and robust fluid transport in a fully decentralized manner.

	From a computing point of view such properties are non-trivial and highly desirable. What \P seems to produce and maintain naturally is an (approximate) solution to the problem of distributing resources in a dynamically changing planar graph. This is an interesting and complex transport problem with various conceivable practical applications, particularly in the domain of operations research. As a result, it is desirable to model the behavior of \P with the goal of developing algorithms for this problem. Ideally such an algorithm would have the following properties:

	\begin{itemize}
		\item The algorithm maintains a dynamic circulation of flow including flow reversals mimicking the flows observed in live \P. Since resources are transported with the flow, they go wherever the flow reaches.
		\item The algorithm is robust against changes in topology. Neither natural changes of network topology nor accidental disconnection of veins renders \P in a state from which it cannot recover. The algorithm should share this quality.
		\item The algorithm is distributed and requires no central control. As a result, complex global coordination of any sort must emerge from local interactions.
		\item The algorithm has a degree of efficiency. Based on the assumption that a certain degree of efficiency is necessary for \P to survive, one may hope that models and algorithms mimicking the organism, inherit this efficiency at least to some extent.
	\end{itemize}

	Paving the way towards such an algorithm constitutes the main motivation of this thesis. In it, we seek to contribute to the field of natural computing with \P through the use of experimental and theoretical methods in an interdisciplinary spirit.

	Ultimately, we aim to find a model which yields a circulating flow including flow reversals for individual edges as displayed by \P. Ideally, this model covers the properties listed above and can be used to derive an efficient algorithm which mimics the way \P distributes resources all across its network of veins by means of peristaltic pumping. 

	The past has shown that successful modeling of natural phenomena tends to build upon a reliable body of experimental data and results concerning the same. The reason for this is twofold: First experimental results help build intuition which is crucial in the formulation of models. Second, experimental data can be used after models and/or algorithms have been derived to assess the degree to which they resemble the workings of the natural phenomenon in question. Thus, before engaging in attempts to model the flows observed in \P, this thesis looks towards experimental work.

	In this context, the oscillator experiments introduced in \Fref{sec:oscillator_experiment} are most promising. They suggests to explore an approach based on the behavior of synchronized coupled oscillators. Experimentally it was shown that dynamic flows, including flow reversals, follow from changing pressure gradients induced by organized thickness oscillations in the plasmodium of \P. Here the experimenters stress in particular that network properties such as the lengths and the thickness of veins critically influence the observed emergent behavior. 

	To gain insights in the dynamics of the flow, one may a) look at the flows directly, or b) study the vein networks themselves. Experimentally it is possible to track the flow of protoplasm through single veins of the organism for a certain amount of time. However, obtaining this information for large vein networks including thousands of veins lies outside of our capabilities and is thus infeasible. Alternatively, one may study the topology of the networks on a large scale. This is idea forms the starting point of this thesis. 

	Thus, we conduct a large number of tailored wet-lab experiments designed to produce images of \P networks. To evaluate and study the obtained experimental data, we develop a software called Network Extraction From Images or \NEFI. It is designed to turn images depicting networks into equivalent graphs. After processing these \P graphs, we obtain a detailed numerical characterization of their properties searching for clues supporting our modeling efforts. 

	Naturally, we want all our results, \ie experimental raw data, processed data as well as our numerical studies, to serve others the same way the serve us. Thus we set up a dedicate repository, the so-called Slime Mold Graph Repository or \SMGR, designed to further the reuse and exchange of \P related data with a focus on slime mold graphs. All results presented in this thesis are readily available to the natural computing community and everyone who is interested online at \href{http://smgr.mpi-inf.mpg.de}{http://smgr.mpi-inf.mpg.de} and \href{http://nefi.mpi-inf.mpg.de/}{http://nefi.mpi-inf.mpg.de/}.

	This thesis documents our studies of the networks formed by \Pp on the road towards novel distributed natural computing approaches inspired by it. In this thesis, we set out on an ambitious interdisciplinary journey with an uncertain destination.

	The remainder of this thesis is structured as a sequence of self-contained chapters supported by relevant appendices. They can be read in order or independently:\vspace{\baselineskip}

	In \Fref{chap:nefi} we introduce \NEFI and the problem of extracting networks or graphs from images. Network extraction remains a vital precursor to network analysis and the key to making raw data from wet-lab experiments tractable. \Fref{chap:smgr} describes the experiments we conducted with live \P. We discuss the experimental setup as well as the data we obtain. Furthermore we introduce the \SMGR, a publicly available data repository intended to promote the distribution and reuse of slime mold related data and results. In \Fref{chap:analysis} we focus on the actual analysis of the data we obtained earlier. Here we present a systematic study of several graph observables characterizing the vein networks formed by \P. Finally, in \Fref{chap:model} we present our efforts of modeling the complex flow patterns known to occur in said networks through the use of electric elements. In particular, we study the properties of circuits formed by these elements combining analytic and numerical methods. Furthermore, we discuss the potential of our approach and its suitability for developing natural computing algorithms inspired by the behavior of \P. %
	% In this regard, we announce upcoming results, which unfortunately did not make it in time to appear in this thesis. 
	We close with a general discussion and concluding remarks regarding the tools, services and results presented. In particular we offer suggestions for potential further research.










% reliable experimental data that describes a certain phenomenon and acts as a guide to the modeling process
% depending on the phenomenon under consideration this can require dedicated experimental methods and data processing approaches.
% the modelling process itself, where the challenge is to look past the intricate bio-physical processes that are observed in \P and find a suitable level of abstraction. It must capture the phenomenon of interest well, yet do away with a large part of the complexity of the organism.
% lastly turning the model into an algorithm.
% this requires a deep understanding of the model itself. The formal analysis of such algorithms belongs to the domain of computer science.

% what we want: get a simple model that captures the way \P uses thickness oscillations to generate pressure differences which generate fluid flow. 

% Why: because we assume that distribution stuff within its whole body is a vital task that \P needs to handle in order to survive. Here the hope is that the method \P uses to solve this task is also efficient. Experimental evidence is given by Alim \etal. The hope is to extract a bio-inspired fully distributed and fault tolerant algorithm capable of distributing goods in a graph.

% why not use the phyarum solver? Because it cannot capture oszillations and the tubes are passive elements. We do take inspiration from the physarum solver because we chose to model \P as an electric system. Studying the properties of the solver helped us realize that they are equivalent.

% why not use jones? because a particle model does not automatically lend itself to a graph abstraction. Might be possible to put it on a graph, however, certainly not natural.

% so what are you going to do?

% * we know that \P needs to solve the problem of efficient nutrient circulation. Alim \etal show evidence that \P does well. Alim \etal model veins as active elements.
% * we look at the oscillator experiment and realize that coupled oscillators replicate the syncronisation behavior. Here we realize that the topology and edge weights determine the behaviour the oscillator
% * we realize that we need to know as much as possible about the topology and the edge weights of \P networks when there is no food around.
% * we do the experiments, we write the software, we get the graphs, we publish everything
% * we look at the modelling of other vein based circulation systems such as the cardio vascular system. Here we learn how to model veins that are actively
% * We decide to model physarum as a system of oscillating electronic elements. In this model the veins are the active part and the nodes have a passive role.

% what is your contribution?

% * supporting the development of natural computing with \P by providing experimental data in form of images and graphs. To this we add NEFI and the SMGR. We hope that by doing so we enable others.

% * working towards the goal of mimicking the way \P distributes fluid all across its body. Ideally, we obtain an effecient, distributed algorithm.
