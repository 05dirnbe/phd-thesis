\documentclass{standalone}
% \usepackage{pgfplots}
% \pgfplotsset{compat=newest}
% \usepackage{amsmath}


\begin{document}
\begin{tikzpicture}
  \begin{axis}[
    axis lines=middle,
    xlabel=$i_p(t)$,
    ylabel={$u^*_p(t)$},
    xmin=-3, xmax=3,
    ymin=-1.5, ymax=1.5,
    xtick=\empty,
    ytick=\empty,
    % ytick={0, 1},
    extra y ticks={1,-1},
    extra y tick labels={$\hat{U}$,$-\hat{U}$},
    function line/.style={
      red,
      thick,
      samples=1000,
    },
    single dot/.style={
      red,
      mark=*,
    },
    empty point/.style={
      only marks,
      mark=*,
      mark options={fill=white, draw=black},
    },
  ]
    % \addplot[function line, domain=\pgfkeysvalueof{/pgfplots/xmin}:0] {-1};
    % \addplot[function line, domain=0:\pgfkeysvalueof{/pgfplots/xmax}] {1};
    % \addplot[single dot] coordinates {(0, 0)};
    % \addplot[empty point] coordinates {(0, -1) (0, 1)};
    % \addplot[draw=blue]{x/2}; % this is the same as x = y*2
    \addplot[function line]{max(min(x,1),-1)}; % this is the same as x = y*2
  \end{axis}
\end{tikzpicture}
\end{document}