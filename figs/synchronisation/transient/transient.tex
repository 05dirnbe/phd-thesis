\documentclass{standalone}
% \usepackage{pgfplots}
% \pgfplotsset{compat=newest}
% \usepackage{amsmath}
% \DeclareMathOperator{\sgn}{sgn}

\begin{document}
\begin{tikzpicture}
  \begin{axis}[
    axis lines=middle,
    xlabel=$t$,
    ylabel={$U$},
    xmin=-0, xmax=6,
    ymin=-0.5, ymax=1.5,
    xtick=\empty,
    ytick=\empty,
    extra x ticks={2},
    extra x tick labels={$t_1$},
    x tick label style={anchor=north west},
    function line/.style={
      red,
      thick,
      samples=100,
    },
    single dot/.style={
      red,
      mark=*,
    },
    empty point/.style={
      only marks,
      mark=*,
      mark options={fill=white, draw=black},
    },
  ]
    % \addplot[function line, domain=\pgfkeysvalueof{/pgfplots/xmin}:0] {-1};
    \addplot[function line, domain=2:\pgfkeysvalueof{/pgfplots/xmax}] {1};
    \addplot[function line, domain=0:2] {0};
    \addplot +[function line, mark=none] coordinates {(2, 0) (2, 1)};
    

    \draw[thick, samples=1000, domain=2:\pgfkeysvalueof{/pgfplots/xmax}] plot({\x},{(1 - (1 * exp((2-\x)*3))}) ;
    \draw[gray, thick, dotted, samples=100, domain=1:\pgfkeysvalueof{/pgfplots/xmax}] plot(\x,{3*\x-6}) ;
   
    \draw[thick, red] (1.5,1) node {$u^*_p(t)$};
    \draw[thick] (3,0.5) node {$u(t)$};
   
    % \addplot[single dot] coordinates {(0, 0)};
    % \addplot[empty point] coordinates {(0, -1) (0, 1)};
  \end{axis}
\end{tikzpicture}
\end{document}